\documentclass[preview, border=5mm]{standalone}

\usepackage[a4paper,margin=10cm]{geometry}

% for the \xintFor***
\usepackage{xinttools} 
\usepackage{tikz}
\usetikzlibrary{calc}


% playing role of infinity (should be < .25\maxdimen)
\def\biglen{297cm}

% define the "half plane" to be clipped (#1 = half the distance between cells)
\tikzset{
  half plane/.style={ to path={
       ($(\tikztostart)!.5!(\tikztotarget)!#1!(\tikztotarget)!\biglen!90:(\tikztotarget)$)
    -- ($(\tikztostart)!.5!(\tikztotarget)!#1!(\tikztotarget)!\biglen!-90:(\tikztotarget)$)
    -- ([turn]0,2*\biglen) -- ([turn]0,2*\biglen) -- cycle}},
  half plane/.default={1pt}
}

% number of random points
\def\cantidadpuntos{30} 
% random points are in [-\maxxy,\maxxy]x[-\maxxy,\maxxy]
\def\maxxy{4}

\begin{document}

  \begin{tikzpicture}

    % generate random points
    % \pgfmathsetseed{1908}

    \def\puntos{}
    \xintFor* #1 in {\xintSeq {1}{\cantidadpuntos}} \do{
      % random x in [-.9\maxxy,.9\maxxy]
      \pgfmathsetmacro{\ptx}{.9*\maxxy*rand} 
      % random x in [-.9\maxxy,.9\maxxy]
      \pgfmathsetmacro{\pty}{.9*\maxxy*rand} 
      % stock the random point
      \edef\puntos{\puntos, (\ptx,\pty)} 
    }

    % draw the points and their cells
    \xintForpair #1#2 in \puntos \do{
      \edef\puntoa{#1,#2}
      \begin{scope}
        \xintForpair \#3#4 in \puntos \do{
          \edef\ptb{#3,#4}

          % check if (#1,#2) == (#3,#4) 
          \ifx\puntoa\ptb\relax 
            \tikzstyle{myclip}=[];
          \else
            \tikzstyle{myclip}=[clip];
          \fi;
          \path[myclip] (#3,#4) to[half plane] (#1,#2);
        }
        % last clip
        \clip (-\maxxy,-\maxxy) rectangle (\maxxy,\maxxy); 

        \pgfmathsetmacro{\randhue}{rnd}
        \definecolor{randcolor}{hsb}{\randhue,.2,.5}
        % fill the cell with random color
        \fill[randcolor] (#1,#2) circle (4*\biglen); 
        % and draw the point
        % \fill[draw=red,very thick] (#1,#2) circle (1.4pt); 
      \end{scope}
    }

    \pgfresetboundingbox
    \draw (-\maxxy,-\maxxy) rectangle (\maxxy,\maxxy);
  \end{tikzpicture}

\end{document}

% Source: https://tex.stackexchange.com/questions/138668/how-to-draw-and-paint-the-voronoi-regions-of-a-series-of-points-using-tikz

% Some comments on the code :
% 
% • Given two points, A and B, the points that are closer to A is a half plane,
%   delimited by the perpendicular bisector, and containing A.
% • So to construct the Voronoi cell of A we can take the intersection of all
%   this half planes when B runs over all other points (different from A).
% • In the code, taking this intersection is done by clipping big rectangles
%   that plays the role of "half planes".
% • I was not able to use \foreach because clipping inside such a loop is not
%   available outside the loop (\foreach creates a group). So I'm overcoming
%   this by using \xintFor.
% 
