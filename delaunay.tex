%\documentclass[preview, border=5mm]{standalone}
\documentclass[a3paper]{slides}

\usepackage[margin=2cm]{geometry}

% for the \xintFor***
\usepackage{xinttools} 
\usepackage{tikz}
\usetikzlibrary{calc,backgrounds}


% playing role of infinity (should be < .25\maxdimen)
\def\bigLen{20cm}

% define the "half plane" to be clipped (#1 = half the distance between cells)
\tikzset{
  half plane/.style={
    to path={
       ($(\tikztostart)!.5!(\tikztotarget)!#1!(\tikztotarget)!\bigLen!90:(\tikztotarget)$)
    -- ($(\tikztostart)!.5!(\tikztotarget)!#1!(\tikztotarget)!\bigLen!-90:(\tikztotarget)$)
    -- ([turn]0,2*\bigLen) -- ([turn]0,2*\bigLen) -- cycle
    }
  },
  half plane/.default={2pt},
}


% \dSat}{ rnd }
\pgfmathsetmacro{\maxX}{ \textwidth  * 0.5 }
\def\maxY{ \maxX * 1.44  }
\def\puntosCantidad{10} 



\begin{document}

\typeout{\maxX}
  \begin{tikzpicture}
    % \pgfmathsetseed{1908}
    % \pgfmathsetmacro{\randSat}{ rnd }

    % Generar Puntos
    \gdef\puntos{}
    \foreach \i in { 0 ,..., \puntosCantidad }{
	    \pgfmathsetmacro{\X}{ 10 * rand }
	\pgfmathsetmacro{\Y}{ 10 * rand }
	\xdef\puntos{\puntos, ( \X, \Y, \i )} 
	  \fill[ draw=black, fill=black ] ( \X, \Y ) circle ( 2pt )node {\i}; 
    }
    % http://www.geom.uiuc.edu/~samuelp/del_project.html






















    % draw the points and their cells
%    \xintForpair #1#2 in \puntos \do{
%      \edef\puntoA{#1,#2}
%
%      \begin{scope}
%        \xintForpair \#3#4 in \puntos \do{
%          \edef\puntoB{#3,#4}
%
%          % check if (#1,#2) == (#3,#4) 
%          \ifx\puntoA\puntoB\relax 
%            \tikzstyle{myClip}=[];
%          \else
%            \tikzstyle{myClip}=[clip];
%          \fi;
%          \path [myClip] (#3,#4) to [half plane] (#1,#2);
%        }
%        % last clip
%        \clip (-\maxX,-\maxY) rectangle (\maxX,\maxY); 
%
%        \pgfmathsetmacro{\randSat}{ rnd }
%        \definecolor{randcolor}{hsb}{ .35, .6, \randSat,  }
%        % fill the cell with random color
%        \fill[randcolor, opacity=.8 ] (#1,#2) circle (14*\bigLen); 
%        % and draw the point
%        \fill[draw=white,fill=white] (#1,#2) circle (2pt); 
%      \end{scope}
%
%    }

    % Borde y Marco

    \draw[draw=white,draw=white,line width=4pt] 
	  ({-\maxX + 20 },{-\maxY + 20 })
	  rectangle 
	  ({\maxX - 20	},{\maxY - 20 });
    \draw[draw=white,draw opacity=.5,line width=1cm] 
	  ({-\maxX },{-\maxY }) 
	  rectangle 
	  ({\maxX },{\maxY });
    \pgfresetboundingbox
    \draw[draw=white] (-\maxX,-\maxY) rectangle (\maxX,\maxY );

  \end{tikzpicture}
\end{document}

