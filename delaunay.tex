%\documentclass[preview, border=5mm]{standalone}
\documentclass[a3paper]{slides}

\usepackage[margin=2cm]{geometry}

% for the \xintFor***
\usepackage{xinttools} 
\usepackage{tikz}
\usetikzlibrary{calc,backgrounds}

%sort
\usepackage{arrayjobx}
\usepackage{multido}
\usepackage{forest}


% playing role of infinity (should be < .25\maxdimen)
\def\bigLen{20cm}

% define the "half plane" to be clipped (#1 = half the distance between cells)
\tikzset{
  half plane/.style={
    to path={
       ($(\tikztostart)!.5!(\tikztotarget)!#1!(\tikztotarget)!\bigLen!90:(\tikztotarget)$)
    -- ($(\tikztostart)!.5!(\tikztotarget)!#1!(\tikztotarget)!\bigLen!-90:(\tikztotarget)$)
    -- ([turn]0,2*\bigLen) -- ([turn]0,2*\bigLen) -- cycle
    }
  },
  half plane/.default={2pt},
}





\makeatletter
% This is based ond the "code 4" of
% https://tex.stackexchange.com/a/273084/4686,
%
% modified to handle pairs (a, b) (c,d)(e, f) (g,h) etc...
%
% acts expandably. Lexicographic order.

\makeatletter

% Here we define the comparison macro for pairs (a,b)
% We assume decimal numbers acceptable to \ifdim tests

\long\def\xintdothis #1#2\xintorthat #3{\fi #1}%
\let\xintorthat \@firstofone

\long\def\@thirdoffour  #1#2#3#4{#3}%
\long\def\@fourthoffour #1#2#3#4{#4}%

\def\IfFirstPairIsGreaterTF #1#2{\@IfFirstPairIsGreaterTF #1,#2,}%

\def\@IfFirstPairIsGreaterTF #1,#2,#3,#4,{%
    \ifdim #1\p@=#3\p@
       \xintdothis{%
         \ifdim #2\p@>#4\p@\expandafter\@firstoftwo
         \else\expandafter\@secondoftwo\fi}\fi
    \ifdim #1\p@>#3\p@\expandafter\@thirdoffour
                      \else\expandafter\@fourthoffour\fi
    \xintorthat{}%
}%

% not needed for numerical inputs
% \catcode`! 3
% \catcode`? 3

% Here there is a very strange \romannumeral0\romannumeral0, this is
% due to some convoluted scheme to avoid double spaces or no spaces
% in between coordinate pairs. Trust me.
\def\QSpairs {\romannumeral0\romannumeral0\qspairs }%
% first we check if empty list
\def\qspairs   #1{\expandafter\qspairs@a\romannumeral-`0#1(!)(?)}%
\def\qspairs@a #1(#2{\ifx!#2\expandafter\qspairs@abort\else
                        \expandafter\qspairs@b\fi (#2}%
\edef\qspairs@abort #1(?){\space\space}%
%
% we check if empty of single and if not pick up the first as Pivot:
\def\qspairs@b #1(#2)#3(#4){\ifx?#4\xintdothis\qspairs@empty\fi
                   \ifx!#4\xintdothis\qspairs@single\fi
                   \xintorthat \qspairs@separate {}{}{#2}(#4)}%
\def\qspairs@empty  #1(?){ }%
\edef\qspairs@single #1#2#3#4(?){(#3)}%
\def\qspairs@separate #1#2#3#4(#5){%
    \ifx!#5\expandafter\qspairs@separate@done\fi
    \IfFirstPairIsGreaterTF {#5}{#3}%
          \qspairs@separate@appendtogreater
          \qspairs@separate@appendtosmaller {#5}{#1}{#2}{#3}
}%
%
\def\qspairs@separate@appendtogreater #1#2{\qspairs@separate {#2 (#1)}}%
\def\qspairs@separate@appendtosmaller #1#2#3{\qspairs@separate {#2}{#3 (#1)}}%
%
\def\qspairs@separate@done\IfFirstPairIsGreaterTF #1#2%
    \qspairs@separate@appendtogreater
    \qspairs@separate@appendtosmaller #3#4#5#6(?)%
{%
    \expandafter\qspairs@f\expandafter
     {\romannumeral0\qspairs@b #4(!)(?)}{\qspairs@b #5(!)(?)}{ (#2)}%
}%
%
\def\qspairs@f #1#2#3{#2#3#1}%
%
% \catcode`! 12
% \catcode`? 12

\makeatother

\begin{document}

\pgfmathsetmacro{\maxX}{ \textwidth }
\def\maxY{ \maxX * 2 }
\def\puntosCantidad{10} 

  \begin{tikzpicture}
    % \pgfmathsetseed{1908}
    % \pgfmathsetmacro{\randSat}{ rnd }

    % Generar Puntos
    \gdef\puntos{}
    \foreach \i in { 0 ,..., \puntosCantidad }{
        \pgfmathsetmacro{\X}{ 10 * rand }
	\pgfmathsetmacro{\Y}{ 1.4 * 10 * rand }
	  %\xdef\puntos{\puntos,  \X/\Y/\i} 
	  \xdef\puntos{\puntos(\X,\Y),} 
    }

    % http://www.geom.uiuc.edu/~samuelp/del_project.html


    % Sort points by its X value
    
	  \message{UNsorted: \puntos}
	  \gdef\pps{\QSpairs{\puntos}}
	  \message{SORTED: \pps}

    \foreach \p in \pps{
        %\draw (\p) circle ( 2pt ) node {iii}; 
        \message{p: \p}
    }

















    % draw the points and their cells
%    \xintForpair #1#2 in \puntos \do{
%      \edef\puntoA{#1,#2}
%
%      \begin{scope}
%        \xintForpair \#3#4 in \puntos \do{
%          \edef\puntoB{#3,#4}
%
%          % check if (#1,#2) == (#3,#4) 
%          \ifx\puntoA\puntoB\relax 
%            \tikzstyle{myClip}=[];
%          \else
%            \tikzstyle{myClip}=[clip];
%          \fi;
%          \path [myClip] (#3,#4) to [half plane] (#1,#2);
%        }
%        % last clip
%        \clip (-\maxX,-\maxY) rectangle (\maxX,\maxY); 
%
%        \pgfmathsetmacro{\randSat}{ rnd }
%        \definecolor{randcolor}{hsb}{ .35, .6, \randSat,  }
%        % fill the cell with random color
%        \fill[randcolor, opacity=.8 ] (#1,#2) circle (14*\bigLen); 
%        % and draw the point
%        \fill[draw=white,fill=white] (#1,#2) circle (2pt); 
%      \end{scope}
%
%    }

    % Borde y Marco

%    \draw[draw=white,draw=white,line width=4pt] 
%	  ({-\maxX + 20 },{-\maxY + 20 })
%	  rectangle 
%	  ({\maxX - 20	},{\maxY - 20 });
%    \draw[draw=white,draw opacity=.5,line width=1cm] 
%	  ({-\maxX },{-\maxY }) 
%	  rectangle 
%	  ({\maxX },{\maxY });
%    \pgfresetboundingbox
%    \draw[draw=white] (-\maxX,-\maxY) rectangle (\maxX,\maxY );

  \end{tikzpicture}
\end{document}

